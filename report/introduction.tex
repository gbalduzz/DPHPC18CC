

\mypar {Motivation}
In computer science problems often get model as graphs and therefore graph algorithms are ubiquitous. One of these graph problems is the task of finding the connected components in a graph. It is a well understood problem in graph theory with a variety of applicable domains. Computer vision tasks such as pattern recognition and image segmentation \cite{683775} can make use of connected component \cite{Wilson:2006:RCV:1166253.1166292}. Other fields are medical imaging \cite{UDUPA1990355} and image processing \cite{Ambrosio2001}. We will not discuss the related problem of strongly connected components.

As already mentioned this problem is well-studied both sequentially and in parallel. The first sequential algorithm goes back to \cite{Hopcroft}. A few parallel approaches would be \cite{MANOHAR1989133}\cite{Han:1990:EFP:79147.214077} and recently \cite{comm_avoiding} where they used a communication-avoiding approach. A communication-avoiding algorithm uses asymptotically less communication. By doing so \cite{comm_avoiding} sacrifice some efficiency in the computation as the root node does most of the work. We wanted to improve on this by also introducing a distributed computation based on hooking \cite{article} and ignoring the communication part. In a additional step we distributed the list of edges evenly among different MPI proccesses. This allows us to outperform the communication avoiding approach especially on denser graphs. Our approach perfroms significantly worse on a small amount of nodes (n$\leq$ 5) but as we increase the total number of cores the benefits of our algorithms starts to show.

