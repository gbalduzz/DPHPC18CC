

\mypar{Motivation}
Problems in computer science are often modeled as graphs. Therefore, graph algorithms are
ubiquitous. One of these graph problems is finding connected components. It is a well understood
problem in graph theory with a variety of applicable domains. Computer vision tasks, such as
pattern recognition and image segmentation \cite{683775} can make use of connected components
\cite{Wilson:2006:RCV:1166253.1166292}. Other fields are medical imaging \cite{UDUPA1990355} and
image processing \cite{Ambrosio2001}. The related problem of strongly connected components will not
be discussed in this paper.

\mypar{Related Work}
The first sequential algorithm to solve the connected components problem goes back to
\cite{Hopcroft}. Parallel approaches were presented in \cite{MANOHAR1989133} and
\cite{Han:1990:EFP:79147.214077}. Recently a communication-avoiding algorithm was published \cite{comm_avoiding}.
This algorithm uses asymptotically less communication, but sacrifices some computational
efficiency as the root node does most of the work. In this paper we present an algorithm which
distributes the work while still avoiding as much communication as possible. This is achieved by
distributing the list of edges evenly among different MPI ranks. These then locally compute their
corresponding connected components which are represented as a forest. In a next step the algorithm
reduces these forests in a binary manner. Two MPI ranks compare and merge their results and then
compress them. This step is repeated until the final result is propagated to the root process.

%TODO: maybe speak about the distributed version


