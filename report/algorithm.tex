\mypar{Connected components}
%Precisely define sorting problem you consider.
%\mypar{Sorting algorithms}
%Explain the algorithm you use including their costs.
%
%As an aside, don't talk about "the complexity of the algorithm.'' It's incorrect,
%problems have a complexity, not algorithms.

For an unordered graph $G=(V,E)$, the connected components are the ensemble
of connected subgraphs, where connected means that for any two  vertices, there exist a path along the
 edges connecting them.
The straightforward algorithm to find them is to perform either a breath or depth first search from a starting
random vertex in $V$, and give the same label to all the touched vertices. Then repeat the search from an unlabelled vertex
until there are no more left.
This has a cost in terms of memory accesses of $\Theta(\abs{E} + \abs{V})$, which turns to be optimal \cite{Hopcroft}.

\section{Proposed Algorithm}\label{sec:yourmethod}
%Now comes the ``beef'' of the report, where you explain what you
%did. Again, organize it in paragraphs with titles. As in every section
%you start with a very brief overview of the section.
%
%In this section, structure is very important so one can follow the technical content.
%
%Mention and cite any external resources that you used including libraries or other code.

Unfortunately this algorithm does not parallelize straightforwardly. Instead we firstly implemented
an algorithm proposed by Uzi Vishkin \cite{PCompPaper} and later described in a class by
Pavel Tvrdik \cite{PCompClass}. This algorithm casts the problem in terms of the generation of a
forest, where the vertices of the same connected component belong to the same three, and its root
can be used as the representative. We define a star as a three of height one, a singleton a tree
with a single element, and use the variables $n=\abs{V}$ and $m=\abs{E}$.
His algorithm can be summarized as:

\begin{algorithm}%[H]
    \caption{Pavel Tvrdik's Connected components}
    \label{algorithm:cc1}
    \begin{algorithmic}[1]
        \Procedure{Hook}{$i, j$}
          \State $p[p[i]] = p[j]$
        \EndProcedure
        \Procedure{connectedComponents}{$n, \text{edges}$}
          \State $p[i] = i \quad \forall i \in \{1,\cdots, n\}$. \Comment{Initialize a list of parents.}
        \While{Elements of $p$ are changed.}
        \For{$\left<i, j\right> \in \text{edges}$} \Comment{Execute in parallel.}
          \State  \kif $i\ge j$ \kthen Hook($i, j$)
          \State  \kif $\text{isSingleton}(i)$ \kthen Hook($i, j$)
        \EndFor
        \For{$\left<i, j\right> \in \text{edges}$} \Comment{Execute in parallel.}
          \State  \kif isStar(i) \kand $i \neq j$ \kthen Hook($i, j$)
        \EndFor
        \State $p[i] = \text{root}(i) \quad \forall i \in \{1,\cdots, n\}$ \Comment{Compress the forest in parallel.}
        \EndWhile
        \EndProcedure
   \end{algorithmic}
\end{algorithm}
We defer to \cite{PCompClass} for a proof of correctness.

After implementing this algorithm we found advantageous to remove the constraint
that only singletons and stars can be hooked to another vertex, so that only a single pass through
the edge list is required. Extra care is then required during parallel execution: as each vertex has only one outgoing
connection, we need to avoid that a process overwrites a connection that has been formed by another one.
We therefore need to grow our forest with the following rules:

\begin{enumerate}
    \item A hook must originate from a vertex id higher than the destination.
    \item All edges must generate a connection between the relative vertices, or vertices at an higher level in their three.
    \item A hook must originate from a vertex that is currently the root of a tree.
\end{enumerate}

The intuitive proof of correctness follows: rule $1$ means that the graph gebnerated by the
hooks generate a directed graph with no cycles and with at most a single outgoing connection, therefore it must be a forest.
Rule $2$ and $3$ enforce that after processing an edge between two nodes,
they belong to the same tree, and rule $3$ guarantees that this connection can not be broken by a different edge.
At the end of the algorithm, by following the connections from each vertex to the root, we can find a representative for each connected component.

To implement rule $3$ in a multithreaded environment, we use an atomic compare and swap.
We compare the parent of the hook's origin with its id, if they match it means the vertex is still a root and we
hook we hook it to its destination. It does not matter for correctness if the destination is a root, but we try withouth enforcing to
hook to a root to minimize the three height.
We found empirically that using \verb|std::atomic_compare_exchange_weak|,
compared to \verb|std::atomic_compare_exchange_strong| offers better performance, as we anyway need to loop until a hook is successful.


In pseudocode our algorithm is:

\begin{algorithm}%[H]
    \caption{Single pass connected component.}
    \label{algorithm:cc2}
    \begin{algorithmic}[1]
        \Procedure{connectedComponents}{$n, \text{edges}$}
        \State $p[i] = i \quad \forall i \in \{1,\cdots, n\}$. %\Comment{Initialize a list of parents.}
        \For{$\left<i, j\right> \in \text{edges}$} \Comment{Execute in parallel.}
        \While{hook is not successful.}
                \State from = max(root(i), root(j))
                \State to = mint(root(i), root(j))
                \State  atomicHook(from, to)
        \EndWhile
        \State  \kif !isRoot(i) \kthen p[i] = root(i) \label{algorithm:step:skip_connection}
        \State  \kif !isRoot(j) \kthen p[j] = root(j)
        \EndFor
        \State $p[i] = \text{root}(i) \quad \forall i \in \{1,\cdots, n\}$ \Comment{Compress the forest in parallel.}
        \EndProcedure
    \end{algorithmic}
\end{algorithm}

While the step \ref{algorithm:step:skip_connection} is not necessary for correctness, we found that
reusing the already computed vertex's representative leads to a smaller tree height. This and the parallel compression works and was tested to
be efficient only on architectures such as x86, where writes to 32 or 64-bits variables storing a label are atomic.

The overall cost of the algorithm is $\Theta((n + m)\langle H \rangle)$, where $\langle H \rangle$ is
the average tree height. Theefore $\langle H \rangle = \Theta(1)$ for a subcritical random graph, and on average (relatively to the execution order of the loop) $\langle H \rangle = \Theta(\log(n))$
for a supercritical one \cite{RandomGraph}.

\mypar{Multiple compute nodes}
Algorithm \label{algorithm:cc2} works only on a single compute node with a shared memory model. Moreover it is efficient
only when the graph is relatively sparse so that the chance of a collision between two processors trying to update the same parent is low.
%TODO reference the data.

