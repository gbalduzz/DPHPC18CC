\mypar{Connected components.}
%Precisely define sorting problem you consider.
%\mypar{Sorting algorithms}
%Explain the algorithm you use including their costs.
%
%As an aside, don't talk about "the complexity of the algorithm.'' It's incorrect,
%problems have a complexity, not algorithms.

For an unordered graph $G=(V,E)$, the connected components are the ensemble
of connected subgraphs, where connected means that for any two  vertices, there exist a path along the
 edges connecting them.
The straightforward algorithm to find them is to perform either a breath or depth first search from a starting
random vertex in $V$, and give the same label to all the touched vertices. Then repeat the search from an unlabelled vertex
until there are no more left.
This has a cost in terms of memory accesses of $O(\abs{E} + \abs{V})$, which turns to be optimal \cite{Hopcroft}.

\section{Proposed Algorithm}\label{sec:yourmethod}
%Now comes the ``beef'' of the report, where you explain what you
%did. Again, organize it in paragraphs with titles. As in every section
%you start with a very brief overview of the section.
%
%In this section, structure is very important so one can follow the technical content.
%
%Mention and cite any external resources that you used including libraries or other code.

Unfortunately this algorithm does not paralallize straightforwardly. Pavel Tvrdik \cite{PCompClass}
proposed to cast the problem in terms of the generation of a forest, where the vertices of the same connected comonent belong
to the same three, and its root can be used as the representative. Defining a star as a three of height one,
his algorithm can be summarized as:

% TODO
\begin{algorithm}
    \caption{Connected components 1.}
    \label{algorithm:cc1}
    \begin{algorithmic}[1]
        %\Procedure{DCA} %\Comment{The g.c.d. of a and b}
        \State do 1.
        \State do 2.
        %\While{$\norm{\Scg{n} - \Scg{n-1}} < \epsilon$} %\Comment{Iterate untill convergence is reached.}
%        \EndWhile
        %\State \textbf{return} $\Sdca$ %\Comment{foo}
        %\EndProcedure
    \end{algorithmic}
\end{algorithm}
